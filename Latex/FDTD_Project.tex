\documentclass{article}

% The margin size has been adjusted to fit the headers.
\usepackage[lmargin=0.5in, rmargin=0.5in, tmargin=0.75in, bmargin=0.75in]{geometry}

% Required for inserting images
\usepackage{graphicx} 
\graphicspath{{./Images/}}

% The next 2 packages allow more math symbols like therefore and because.
\usepackage{amsmath}
\usepackage{esint}   % Allows double and triple closed integrals.
\usepackage{amssymb}

% This package causes the following cautions: "Not defining \perthousand" and "Not defining \micro."
\usepackage{gensymb} 

% The following 3 packages make the quotation marks symbols populate in the correct direction:
\usepackage [english]{babel}
\usepackage [autostyle, english = american]{csquotes}
\MakeOuterQuote{"}

% Allows for the writing out of the division of polynomials
\usepackage{polynom}

% Allows terms to have a slash through them when showing every step of an algebraic process.
\usepackage{cancel}

% Allows the use of headers and footers.
\usepackage{fancyhdr}

% Use for writing code
\usepackage{listings}

% Use for boxing equations in alignment environments
\usepackage{empheq}

% Allows the use of letters in an enumerate environment:
\usepackage{enumitem}

% Document preamble:
\title{Finite Difference Method - Time Domain Project\\EERF 6351}
\author{Jonathan Rusnak}
\date{February 2025}


\begin{document}
    
    \maketitle

    \tableofcontents

    \section{Differential Equations}
        Write a system of linear differential equations for Ampere's law and Faraday's law.
        \[
            \nabla\times\overrightarrow{E}
            =
            -
            \frac{\partial \overrightarrow{B}}{\partial t}
        \]
        \[  
            \nabla \times \overrightarrow{H}
            =
            \overrightarrow{J}
            +
            \frac{\partial\overrightarrow{D}}{\partial t}
        \]
        For Faraday's law, we have:
        \[
            \nabla\times\overrightarrow{E}
            =
            \left|
                \begin{matrix}
                    \hat{i} & \hat{j} & \hat{k}\\
                    \frac{\partial}{\partial x} &\frac{\partial}{\partial y} &\frac{\partial}{\partial z} \\
                    E_x &E_y & E_z
                \end{matrix}
            \right|
            =
            \langle
                \frac{\partial E_z}{\partial y}
                -
                \frac{\partial E_y}{\partial z}
                ,
                \frac{\partial E_x}{\partial z}
                -
                \frac{\partial E_z}{\partial x}
                ,
                \frac{\partial E_y}{\partial x}
                -
                \frac{\partial E_x}{\partial y}
            \rangle
            =
            -
            \langle
                \frac{\partial B_x}{\partial t},
                \frac{\partial B_y}{\partial t},
                \frac{\partial B_z}{\partial t}
            \rangle
        \]
        \begin{align}
            -\frac{\partial B_x}{\partial t}&=\frac{\partial E_z}{\partial y}-\frac{\partial E_y}{\partial z}\\
            -\frac{\partial B_y}{\partial t}&=\frac{\partial E_x}{\partial z}-\frac{\partial E_z}{\partial x}\\
            -\frac{\partial B_z}{\partial t}&=\frac{\partial E_y}{\partial x}-\frac{\partial E_x}{\partial y}
        \end{align}
        \[  
            \nabla \times \overrightarrow{H}
            =
            \overrightarrow{J}
            +
            \frac{\partial\overrightarrow{D}}{\partial t}
        \]
        \[
            \nabla\times\overrightarrow{H}
            =
            \left|
                \begin{matrix}
                    \hat{i} & \hat{j} & \hat{k}\\
                    \frac{\partial}{\partial x} &\frac{\partial}{\partial y} &\frac{\partial}{\partial z} \\
                    H_x &H_y & H_z
                \end{matrix}
            \right|
            =
            \langle
                \frac{\partial H_z}{\partial y}
                -
                \frac{\partial H_y}{\partial z}
                ,
                \frac{\partial H_x}{\partial z}
                -
                \frac{\partial H_z}{\partial x}
                ,
                \frac{\partial H_y}{\partial x}
                -
                \frac{\partial H_x}{\partial y}
            \rangle
            =
            \langle
                \frac{\partial D_x}{\partial t},
                \frac{\partial D_y}{\partial t},
                \frac{\partial D_z}{\partial t}
            \rangle
        \]
        \begin{align}
            \frac{\partial D_x}{\partial t}+\sigma E_x&=\frac{\partial H_z}{\partial y}-\frac{\partial H_y}{\partial z}\\
            \frac{\partial D_y}{\partial t}+\sigma E_y&=\frac{\partial H_x}{\partial z}-\frac{\partial H_z}{\partial x}\\
            \frac{\partial D_z}{\partial t}+\sigma E_z&=\frac{\partial H_y}{\partial x}-\frac{\partial H_x}{\partial y}
        \end{align}
        Changes with respect to $y$ or $z$ are set to zero.
        The electric field in the $z$ direction is also set to zero:
        \begin{align*}
            -\mu\frac{\partial H_x}{\partial t}&=\frac{\partial E_z}{\partial y}-\frac{\partial E_y}{\partial z}\\
            -\mu\frac{\partial H_y}{\partial t}&=\frac{\partial E_x}{\partial z}-\frac{\partial E_z}{\partial x}\\
            -\mu\frac{\partial H_z}{\partial t}&=\frac{\partial E_y}{\partial x}-\frac{\partial E_x}{\partial y}\\
            \frac{\partial D_x}{\partial t}&=\frac{\partial H_z}{\partial y}-\frac{\partial H_y}{\partial z}-\sigma E_x\\
            \frac{\partial D_y}{\partial t}&=\frac{\partial H_x}{\partial z}-\frac{\partial H_z}{\partial x}-\sigma E_y\\
            \frac{\partial D_z}{\partial t}&=\frac{\partial H_y}{\partial x}-\frac{\partial H_x}{\partial y}-\sigma E_z
        \end{align*}
        \begin{align*}
            -\mu\frac{\partial H_x}{\partial t}&=0-0\\
            -\mu\frac{\partial H_y}{\partial t}&=0-\frac{\partial}{\partial x}0\\
            -\mu\frac{\partial H_z}{\partial t}&=\frac{\partial E_y}{\partial x}-0\\
            \epsilon\frac{\partial E_x}{\partial t}&=0-0-\sigma E_x\\
            \epsilon\frac{\partial E_y}{\partial t}&=0-\frac{\partial H_z}{\partial x}-\sigma E_y\\
            \epsilon\frac{\partial }{\partial t}0&=\frac{\partial H_y}{\partial x}-0-\sigma E_z
        \end{align*}
        \begin{align*}
            -\mu\frac{\partial H_x}{\partial t}&=0\\
            -\mu\frac{\partial H_y}{\partial t}&=0\\
            -\mu\frac{\partial H_z}{\partial t}&=\frac{\partial E_y}{\partial x}\\
            \epsilon\frac{\partial E_x}{\partial t}&=-\sigma E_x\\
            \epsilon\frac{\partial E_y}{\partial t}&=-\frac{\partial H_z}{\partial x}-\sigma E_y\\
            0&=\frac{\partial H_y}{\partial x}-0
        \end{align*}
        \begin{align*}
            -\mu\frac{\partial H_z}{\partial t}&=\frac{\partial E_y}{\partial x}\\
            \epsilon\frac{\partial E_y}{\partial t}+\sigma E_y&=-\frac{\partial H_z}{\partial x}
        \end{align*}
        \begin{align*}
            \frac{\partial H_z}{\partial t}&=-\frac{1}{\mu}\frac{\partial E_y}{\partial x}\\
            \frac{\partial E_y}{\partial t}&=-\frac{1}{\epsilon}\left[\frac{\partial H_z}{\partial x}+\sigma E_y\right]
        \end{align*}

    \newpage
    \section{Central Difference Equations}
        \subsection{Faraday's Law}
            \[
                \frac{\partial H_z}{\partial t}
                =
                -\frac{1}{\mu}\frac{\partial E_y}{\partial x}
            \]
            The magnetic field and the electric field will be interleaved in time and space. 
            To make the Yee grid work effectively, the forward difference equations will be used in space, and the central difference equations will be used in time.
            \[
                \frac{H_z^{n+\frac{1}{2}}\left(i+ \frac{1}{2} \right) - H_z^{n-\frac{1}{2}}\left(i+ \frac{1}{2} \right)  }{\Delta t}
                =
                -\frac{1}{u(i)}
                \frac{E_y^n(i+1)-E_y^n(i)}{2\Delta x}
            \]
            Rearranging gives:
            \[
                H_z^{n+\frac{1}{2}}\left(i+ \frac{1}{2} \right)
                =
                H_z^{n-\frac{1}{2}}\left(i+ \frac{1}{2} \right)
                -\frac{\Delta t}{u(i)}
                \frac{E_y^n(i+1)-E_y^n(i)}{2\Delta x}
            \]

        \subsection{Ampere's Law}
            \[
                \frac{\partial E_y}{\partial t}
                =
                -\frac{1}{\epsilon}
                \left[
                    \frac{\partial H_z}{\partial x}
                    +
                    \sigma E_y
                \right]
            \]
            \[
                \frac{E_y^{n+1}(i) - E_y^{n}(i)}{\Delta t}
                =
                -\frac{1}{\epsilon}
                \left[
                    \frac{H_z^{n+\frac{1}{2}}\left( i+\frac{1}{2} \right) -H_z^{n+\frac{1}{2}}\left( i-\frac{1}{2} \right) }{\Delta x}
                    +
                    \sigma(i) E_y^{n}(i)
                \right]
            \]

            \[
                E_y^{n+1}(i) - E_y^{n}(i)
                =
                -\frac{\Delta t}{\epsilon}
                \frac{H_z^{n+\frac{1}{2}}\left( i+\frac{1}{2} \right) -H_z^{n+\frac{1}{2}}\left( i-\frac{1}{2} \right) }{\Delta x}
                -\frac{\Delta t}{\epsilon}
                \sigma(i) E_y^{n}(i)
            \]
            \[
                E_y^{n+1}(i)
                =
                -\frac{\Delta t}{\epsilon}
                \frac{H_z^{n+\frac{1}{2}}\left( i+\frac{1}{2} \right) -H_z^{n+\frac{1}{2}}\left( i-\frac{1}{2} \right) }{\Delta x}
                -\frac{\Delta t}{\epsilon}
                \sigma(i) E_y^{n}(i)
                +
                E_y^{n}(i)
            \]
            \[
                E_y^{n+1}(i)
                =
                -\frac{\Delta t}{\epsilon}
                \frac{H_z^{n+\frac{1}{2}}\left( i+\frac{1}{2} \right) -H_z^{n+\frac{1}{2}}\left( i-\frac{1}{2} \right) }{\Delta x}
                +
                E_y^{n}(i)
                \left[
                    1
                    -
                    \frac{\Delta t\sigma(i)}{\epsilon}
                \right]
            \]
            This gives the final system of recurrence relations:
            \begin{equation}
                E_y^{n+1}(i)
                =
                \left[
                    1
                    -
                    \frac{\Delta t\sigma(i)}{\epsilon(i)}
                \right]
                E_y^{n}(i)
                -
                \frac{\Delta t}{\Delta x\epsilon(i)}
                \left[
                    H_z^{n+\frac{1}{2}}\left( i+\frac{1}{2} \right) -H_z^{n+\frac{1}{2}}\left( i-\frac{1}{2} \right)
                \right]
            \end{equation}
            The above is coupled to the equation below, and the below must be evaluated first:
            \begin{equation}
                H_z^{n+\frac{1}{2}}\left(i+ \frac{1}{2} \right)
                =
                H_z^{n-\frac{1}{2}}\left(i+ \frac{1}{2} \right)
                -\frac{\Delta t}{u(i)}
                \frac{E_y^n(i+1)-E_y^n(i)}{2\Delta x}
            \end{equation}
            
    \section{Programming}
        For the purposes of programming, the equations must be rewritten to be more digestible. Let:
        \[
            A=
            \frac{\Delta t}{2\Delta xu(i)}
        \]
        \[
            B=
            \left[
                1
                -
                \frac{\Delta t\sigma(i)}{\epsilon(i)}
            \right]
        \]
        \[
            C=
            \frac{\Delta t}{\Delta x\epsilon(i)}
        \]
        This gives the following system to program:
        \begin{empheq}{align}
                H_z^{n+\frac{1}{2}}\left(i+ \frac{1}{2} \right)&=H_z^{n-\frac{1}{2}}\left(i+ \frac{1}{2} \right)-A(i)\left( E_y^n(i+1)-E_y^n(i) \right)\\
                E_y^{n+1}(i)&=B(i)E_y^{n}(i)-C(i)\left[H_z^{n+\frac{1}{2}}\left( i+\frac{1}{2} \right) -H_z^{n+\frac{1}{2}}\left( i-\frac{1}{2} \right)\right]
        \end{empheq}

        


\end{document}